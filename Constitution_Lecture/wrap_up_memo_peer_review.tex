\documentclass[aspectratio=169]{beamer}
\usepackage{lmodern}
\usetheme{Madrid}
%\usecolortheme{giantoak}
\newcommand*\oldmacro{}
\let\oldmacro\insertshorttitle
\renewcommand*\insertshorttitle{\oldmacro\hfill\insertframenumber\,/\,\inserttotalframenumber}
\usepackage[framemethod=tikz]{mdframed}

\usepackage{beamerthemesplit}
\usepackage{textpos}
\usepackage{pgf}
%\logo{\pgfputat{\pgfxy(0,-.4)}{\pgfbox[right,base]{\includegraphics[height=1.0cm]{logo.jpg}}}}
%\newcommand{\nologo}{\setbeamertemplate{logo}{}}
\usepackage{booktabs}
\usepackage{graphicx}
\theoremstyle{principle}
\newtheorem*{principle}{Design Principle}


\titlegraphic{\includegraphics[width=1.0\paperwidth]{cool-wind-800px.jpg}}

\title{Amendments}
%\author[Jeremy Kedziora]{Wind Data Science Team\\
%\small{Uptake}}
\date{}

\begin{document}

%@@@@@@@@@@@@@@@@@@@@@@@@@@@@@@@@@@@@@@@@@@@@@@@@@
\begin{frame}
\frametitle{One Nation, Under Money: Two goals for us...}

\begin{itemize}

\item Put the Constitution up against a second evaluation criterion;
\begin{itemize}
\item Is the Constitution sufficiently dynamic?
\item Podcast considers the 1d version of this via the commerce clause;
\item Strategy: compare historical changes in Constitution to pace of chance in society;
\end{itemize}
\bigskip
\bigskip
\item Explore one of the three arenas of policymaking -- the judiciary;
\begin{itemize}
\item Supreme court decisions make policy in two ways:
\begin{itemize}
\item They change statutory law as it stands on the ground;
\item They reshape the policy for making policies;
\end{itemize}
\item They use the same machinery we will: problem statements (need to resolve multiple interpretations) and evaluation criteria (consistency with social contract, downstream effects);
\end{itemize}
\bigskip
\bigskip

\item Like amendments, Supreme Court decisions have the effect of changing the Constitution -- full exploration of rate of change must count both.

\end{itemize}

\end{frame}

%@@@@@@@@@@@@@@@@@@@@@@@@@@@@@@@@@@@@@@@@@@@@@@@@@
\begin{frame}
\frametitle{Policy Memo Structure}

TO: Political Science 272 Students, Fall Term 2021\\
FROM: Jeremy Kedziora\\
DATE: 22 September, 2021\\
RE: Policy Memo Structure\\
\begin{itemize}

\item Executive Summary;
\bigskip
\item Background Summary;
\bigskip
\item Issue Analysis;
\bigskip
\item Policy Options;
\bigskip
\item Recommendation;
\bigskip
\item Conclusion.

\end{itemize}

\end{frame}

%@@@@@@@@@@@@@@@@@@@@@@@@@@@@@@@@@@@@@@@@@@@@@@@@@
\begin{frame}
\frametitle{Executive Summary}

\begin{itemize}

\item State the problem, preview the organization of your memo, summarize the recommendation;
\bigskip
\bigskip
\bigskip
\item Practice brevity, brevity, brevity;
\bigskip
\bigskip
\bigskip
\item Can be done in 3--4 sentences.

\end{itemize}

\end{frame}

%@@@@@@@@@@@@@@@@@@@@@@@@@@@@@@@@@@@@@@@@@@@@@@@@@
\begin{frame}
\frametitle{Background Summary}

\begin{itemize}

\item Provide a brief history, context, and magnitude of the problem; 
\bigskip
\bigskip
\bigskip
\item Rubric:
\begin{itemize}
\item Clearly state the problem (2 pts);
\item Provide a brief history, context of problem, and magnitude of problem (1 pts);
\item Establish credibility of research via robust sourcing (1 pt).
\end{itemize}
\end{itemize}

\end{frame}

%@@@@@@@@@@@@@@@@@@@@@@@@@@@@@@@@@@@@@@@@@@@@@@@@@
\begin{frame}
\frametitle{Issue Analysis}

\begin{itemize}

\item Analyze the problem and establish its causes:
\begin{itemize} 
\item How did it come about? 
\item Why is it important to solve? 
\item Who are the stakeholders, and what positions have they taken?  
\item What are the likely consequences of maintaining the status quo and of making change?
\end{itemize}
\bigskip
\bigskip
\item If useful, use charts, tables, bullet points – whatever concisely lays out how the problem is caused etc.;
\bigskip
\bigskip
\item Rubric:
\begin{itemize}
\item Establish what causes the problem (2 pts);
\item Consequences of maintaining the status quo (1 pt);
\item Stakeholder analysis (1 pt);
\end{itemize}
\end{itemize}

\end{frame}

%@@@@@@@@@@@@@@@@@@@@@@@@@@@@@@@@@@@@@@@@@@@@@@@@@
\begin{frame}
\frametitle{Policy Options}

\begin{itemize}

\item Establish evaluation criteria and present the top policy options:
\begin{itemize} 
\item Include concise discussion of pros and cons for each; 
\item No straw people -- the options not recommended should be the next best;
\end{itemize}
\bigskip
\bigskip
\item If useful, use charts, tables, bullet points – whatever conveys viable options with an assessment of each;
\bigskip
\bigskip
\item Rubric:
\begin{itemize}
\item Establish evaluation criteria (2 pts);
\item Clearly list two to three options (2 pts);
\item Articulate pros and cons for each option (1 pt);
\item Detail implementation plans (1 pt);
\item Establish policymaker authority to take action (1 pt);
\end{itemize}
\end{itemize}

\end{frame}

%@@@@@@@@@@@@@@@@@@@@@@@@@@@@@@@@@@@@@@@@@@@@@@@@@
\begin{frame}
\frametitle{Recommendation}

\begin{itemize}

\item Argue for the option that you believe presents the best strategy. 
 
\begin{itemize} 
\item Discuss strategy of implementation, e.g. are you proposing a multi-step strategy/solution? 
\item Highlight potential unknowns or contingencies that might alter the recommendation;
\end{itemize}
\bigskip
\bigskip
\item This is the heart of your memo!
\bigskip
\bigskip
\item Rubric:
\begin{itemize}
\item Clearly recommend exactly one option off the list of options (2 pts);
\item Link option pro/cons to basis for analysis/decision criteria (2 pts);
\item Identify next steps to ensure effectiveness of recommended policy (1 pt);
\end{itemize}
\end{itemize}

\end{frame}

%@@@@@@@@@@@@@@@@@@@@@@@@@@@@@@@@@@@@@@@@@@@@@@@@@
\begin{frame}
\frametitle{Conclusion}

\begin{itemize}

\item Should not simply rehash of what you’ve written:
\begin{itemize} 
\item Summarize your recommendation (one sentence);
\item Leave your audience with an impetus to act on the problem by adopting your strategy;
\end{itemize}
\bigskip
\bigskip
\item This is your only opportunity for impassioned rhetoric!
\bigskip
\bigskip
\item Rubric:
\begin{itemize}
\item Meet page limit and organizational structure (2 pts);
\item Consistent citations (2 pts);
\item Grammar/writing clarity (1 pt);
\end{itemize}
\end{itemize}

\end{frame}

%@@@@@@@@@@@@@@@@@@@@@@@@@@@@@@@@@@@@@@@@@@@@@@@@@
\begin{frame}
\frametitle{Peer Review}

\begin{itemize}

\item Peer review = evaluation of work by people with similar levels of expertise as those who produced the work;
\begin{itemize}
\item Very old -- examples from 9th and 17th centuries;
\item Flawed -- low sample size, closed mindedness, high outcome variance;
\end{itemize}
\bigskip
\item For us, peer reviewed source $=$ published at an academic journal or by a university press;
\bigskip
\item Policy memos are NOT peer reviewed;
\bigskip
\item Government research is `sort of' peer reviewed;
\bigskip
\item Published in scientific journal $>$ government research $>$ everything else.
\bigskip
\item Process: faculty websites, e.g. https://limnology.wisc.edu/, google scholar, etc.
\end{itemize}

\end{frame}



\end{document}