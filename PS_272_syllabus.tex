%\documentclass[10pt,twocolumn]{article}
\documentclass[12pt]{article}

\title{Political Science 272: Introduction to Public Policy}
\author{Jeremy Kedziora}
\usepackage{amsmath,amssymb,graphicx,amsthm}
\usepackage{setspace}
\usepackage{palatino}
\usepackage{mathpazo}
\usepackage{url}
\usepackage{hyperref}

\usepackage[left=1in,top=1in,right=1in,bottom=1in,nohead]{geometry}
%\usepackage[left=0.5in,top=0.75in,right=0.5in,bottom=1in,nohead]{geometry}

\singlespacing
%\author{Jeremy Kedziora}

\newtheorem{theorem}{Theorem}[section]
\newtheorem{lemma}[theorem]{Lemma}
\newtheorem{proposition}[theorem]{Proposition}
\newtheorem{corollary}[theorem]{Corollary}
\newtheorem{definition}[]{Definition}
\newtheorem{assumption}[]{Assumption}
\newtheorem*{example}{Example 1}

\newenvironment{result}[1][Result]{\begin{trivlist}
\item[\hskip \labelsep {\bfseries #1}]}{\end{trivlist}}

\newenvironment{remark}[1][Remark]{\begin{trivlist}
\item[\hskip \labelsep {\bfseries #1}]}{\end{trivlist}}
\newcommand{\argmax}{\operatornamewithlimits{argmax}}
\newcommand{\argmin}{\operatornamewithlimits{argmin}}

\newcommand{\pderiv}[2]{\frac{\partial #1}{\partial #2}}
\newcommand{\secondpderiv}[3]{\frac{\partial^2 #1}{\partial #2\partial #3}}

\begin{document}
\noindent\Large \textbf{Political Science 272: Introduction to Public Policy}\\
\\
\normalsize Jeremy Kedziora\\
%Senior Lecturer, Department of Political Science\\
jtkedziora@wisc.edu\\
(608)239-8796\\
\\
M-W 4:00 p.m.-5:15 p.m. Grainger 2120\\
Office hours: 5:15-5:45 MW, Friday 11:30-12:30 virtually, or by appointment\\

\noindent \Large \textbf{Course Overview}\normalsize\\
\\
Public policy is defined in many ways. I think of public policy as the government’s statement of what it intends to do to address problems that cannot, or should not, be solved in a purely private way.  In this problem solving sense public policy is made at the international, national, state, regional, county, city, and even more local levels.  Some public policies address large problems (e.g., international policies to address climate change).  Other public policies focus on highly localized solutions to problems (e.g. local mask wearing ordinances during COVID-19).  Policy initiatives operating at very different levels of scale share a common impetus: Governmental or quasi-governmental entities have concluded that a problem exists, and that it requires the allocation of public resources to mitigate, or prevent, the resulting harms.\\

\noindent Learning about public policy—what it is, the legal frameworks within which it is made, the tools available to policymakers, the policymaking process, and how to evaluate public policies—will strengthen your writing, analytical, research, and advocacy skills, and will inform your participation in our society.\\

\noindent Finally, why is this class in the political science department and how does it differ from other political topics?  Political debates define problems, goals, and agendas, all of which are the fundamental currency of actual policy making.  At the same time, extant policy defines the political agenda and shapes the political environment.  Therefore, policy emerges from and shapes politics.  Beyond that, to achieve any goal through policy, evidence about the effects of different policy tools is indispensable. Evidence must come from sources that your audience will trust. Good evidence ought to be convincing to reasonable opponents of one’s policy goals. Good arguments clarify your logic, even to those who may oppose your goals.\\

\noindent \Large \textbf{Required Texts}\normalsize\\

\noindent There are three texts for this course.  I suggest ordering e-versions of the books to maximize cost savings and convenience.  Both Egan and Birkland are available on Amazon or in the apple ebook store.%Three texts are on reserve at College Library.  Any edition is fine, but more recent editions have more contemporary examples. I will post other readings on Canvass or the class website.

\begin{itemize}
\item Dan Egan, \textit{The Death and Life of the Great Lakes}
\item Thomas A. Birkland, \textit{An Introduction to the Policy Process: Theories, Concepts, and Models of Public Policy Making}. Any Edition (Armonk, NY: M.E. Sharpe).
\item Deborah Stone, \textit{Policy Paradox: The Art of Political Decision Making}. Any Edition (New York: W.W. Norton and Co.).
\end{itemize}

\noindent \Large \textbf{Learning Objectives}\normalsize\\
\begin{itemize}
\item understand the provisions of the United States Constitution most related to the development and execution of public policy in the United States:  Delegation of powers; separation of powers; federalism; and the allocation of powers within the states.
\item understand the difference between federalism and the allocation of authorities in the State of Wisconsin between state, county, and municipal governments.
\item learn to define and frame problems as an essential first step in the development of public policy.
\item learn to develop, analyze, and advocate policy alternatives.
\item learn how to write a policy memo.
\item learn about executive, administrative, legislative, judicial, and other governmental authorities and their respective roles in making public policy.
\item understand the role of politics in policy development.
\item learn various approaches to policy analysis.
\item learn about various tools and functions available to public policy-makers.
\item learn various approaches to evaluate the success of public policy initiatives.
\item learn about the role of norms and values in public policy formulation.
\end{itemize}

\noindent \Large \textbf{Course Requirements and Basis for Assessment}\normalsize\\
\\
Credit hours will be earned by attending two classes of 1.25 hours each, reading and preparing written work outside of class for 6 to 9 hours per week, submitting three policy memos, and taking a final exam.\\

\noindent Engaging with the weekly material is required.  If you are going to miss class, please notify me by email in advance, if possible.  Written assignments must be submitted on time.  Grades will be reduced by 2 points per 24-hour period for which the assignment is late (this is approximately one full grade).\\

\noindent \textbf{Participation} (15 X 1\% = 15\%).
Participation posts to Canvas are due by midnight every Tuesday (except week 2, where they are due by Thursday).
\begin{itemize}
\item Weeks 2 and 3: Ask a question: post a question from the readings of least 100 words AND respond to at least one other post
\item Week 4: Gather evidence: Find a policy-relevant peer-review research paper and post at least 100 words about it on Canvas
\item Week 5: Ask a question about how our local government works (for example, about the relative authority of the Dane County Executive vs. County Board of Supervisors)
\item Week 6: Engage in policymaking: Write to a public official or agency. Post at least 100 words about it and a link to the opportunity on Canvas (e.g. Comment on a proposed federal agency policy, Comment on a proposed state agency policy, Recommend a course of action to one of your elected representatives )
\item Week 7: Engage others in policymaking: Write at least 100 words about why it is important to engage in a particular policy process and link to the opportunity on Canvass OR Write no more 240 characters (plus a link to the opportunity) on why people should engage that is shared by at least 5 other people–post a link or screenshot to Canvas
\item Weeks 8-14: Choose one of the above options. Early posts set the agenda!
\item Every week: Attend lectures or let me know ahead of time if you must miss.
\end{itemize}

\noindent \textbf{Policy memoranda} (3 X 25\% = 75\%).
You will write three policy memos to public officials following the memo template exactly.  Example memos are on Canvas. %In addition to the peer-reviewed research you have shared, I compiled sources of authority and current initiatives in Dane County.
\begin{itemize}
\item Memo \#1 topic announced September 29, due at midnight October 15.

\item Memo \#2 topic announced October 22, due by at midnight November 12.

\item Memo \#3 topic announced November 20, due at midnight December 10.
\end{itemize}


\noindent \textbf{Exam} (10\%).  The exam is scheduled for Dec 20, 2021 from 7:45 AM - 9:45 AM.  

\begin{itemize}
\item It will cover the entire course.  Please do not take this class if you cannot be present for the final exam.
\item The exam evaluates if you did the reading thoughtfully, paid close attention in lectures, and asked when you did not understand a term or concept (raise your hand or email me anytime).
\end{itemize}

%If we aim to talk the talk, we must know the key terms. I will select 20 to 30 key terms from readings and lectures. You will answer several true/false questions about each term’s use in the policy context, give an example, and, in a few sentences, explain how it helps us understand the policy process.\\

\noindent \Large \textbf{Reading/Topic Schedule}\normalsize\\

\noindent You are expected to do all assigned readings for each week before Monday’s class. I will call on students during class.  Each week, we will read some original research and portions of a textbook for a broader context.

\begin{itemize}
%@@@@@@@@@@@@@@@@@@@@@@@@@@@@@@@@@@@@@@@@
\item Week 1 (8 September): Course Overview

%@@@@@@@@@@@@@@@@@@@@@@@@@@@@@@@@@@@@@@@@
\item Week 2 (13 September, 15 September): The Constitution and Federalism
\begin{itemize}
\item Context: \href{https://constitutioncenter.org/interactive-constitution/the-constitution}{The Constitution}
\item Context: \href{https://www.theatlantic.com/magazine/archive/2020/11/what-if-trump-refuses-concede/616424/}{The Election That Could Break America}
\end{itemize}

%@@@@@@@@@@@@@@@@@@@@@@@@@@@@@@@@@@@@@@@@
\item Week 3 (20 September, 22 September): The Policy Process
\begin{itemize}
\item Research: Egan, Part 1 
\item Listen: \href{https://www.wnycstudios.org/podcasts/radiolabmoreperfect/episodes/one-nation-under-money}{More Perfect, ``One Nation, Under Money"}  (Note: this episode includes a brief mention of sexual assault in the context of the Violence Against Women Act at minute 51. It is not graphic.)
\end{itemize}

%@@@@@@@@@@@@@@@@@@@@@@@@@@@@@@@@@@@@@@@@
\item Week 4 (27 September, 29 September): Institutions %(Guest: Thomas Durkin, Research Librarian)
\begin{itemize}
\item Research: Egan, Part 2
\item Listen: \href{https://fedsoc.org/commentary/podcasts/deep-dive-episode-82-a-preview-of-county-of-maui-hawaii-v-hawaii-wildlife-fund}{The Federalist Society ``A Preview of County of Maui, Hawaii v. Hawaii Wildlife Fund"}
\item Context: Birkland, Chapter 1-2
\end{itemize}

%@@@@@@@@@@@@@@@@@@@@@@@@@@@@@@@@@@@@@@@@
\item Week 5 (4 October, 6 October): Policy Actors and Evidence
\begin{itemize}
\item Research: Egan, Part 3
\item Context: Birkland Part II, Chapters 4-7
\end{itemize}

%@@@@@@@@@@@@@@@@@@@@@@@@@@@@@@@@@@@@@@@@
\item Week 6 (11 October, 13 October): Policy Tools %(Guests: Cap Times reporters Natalie Yahr and Abigail Becker)
\begin{itemize}
\item Listen: \href{https://scholars.org/podcast/moonshots}{Moonshots-Thomas Kalil}
\item Listen: \href{https://scholars.org/podcast/informing-policy}{Informing Policy-Jenni W. Owen}
\item Listen: \href{https://scholars.org/podcast/paying-pollution}{SSN: Paying for Pollution-Leigh Raymond}
%\item Listen: \href{https://omny.fm/shows/madsplainers/when-pop-culture-and-local-government-collide}{When pop culture and local government collide: local government reporters review Parks and Recreation, Sim City, The Simsons, The Wire, and Gilmore Girls}
\item Context: Bardach Appendix B (PDF online)
\end{itemize}
\textbf{Memo 1 due at midnight on October 15}

%@@@@@@@@@@@@@@@@@@@@@@@@@@@@@@@@@@@@@@@@
\item Week 7 (18 October, 20 October): Policy Tools %(Guest: Dane County Executive, Joe Parisi)
\begin{itemize}
\item Context: Birkland Part III, Chapters 8-9
\end{itemize}

%@@@@@@@@@@@@@@@@@@@@@@@@@@@@@@@@@@@@@@@@
\item Week 8 (25 October, 27 October): Theories of the Policy Process
\begin{itemize}
\item Research: \href{https://www.vox.com/polyarchy/2016/6/30/12068490/too-many-lawyers-politics}{There are too many lawyers in politics. Here’s what to do about it.-Lee Drutman}
\item Listen: \href{https://scholars.org/podcast/lawyers-lawyers-and-more-lawyers}{SSN: Lawyers, Lawyers, and More Lawyers-Adam Bonica}
\item Context: Birkland Part IV, Chapter 10
\end{itemize}

%@@@@@@@@@@@@@@@@@@@@@@@@@@@@@@@@@@@@@@@@
\item Week 9 (1 November, 3 November): Politics and Rationality
\begin{itemize}
\item Listen: \href{https://www.npr.org/2011/11/28/142721675/obama-office-alters-more-federal-rules-than-bush}{NPR: Obama Office Alters More Federal Rules Than Bush}
\item Context: Stone, Introduction and Chapter 1
\end{itemize}

%@@@@@@@@@@@@@@@@@@@@@@@@@@@@@@@@@@@@@@@@
\item Week 10 (8 November, 10 November): Policy Goals and Tradeoffs %(Guest: Rep. Katrina Shankland)
\begin{itemize}
\item Listen: \href{https://scholars.org/podcast/death-thousand-cuts}{SSN: Death by a Thousand Cuts}
\item Context: Stone, Part II
\end{itemize}
\textbf{Memo 2 due at midnight November 12}

%@@@@@@@@@@@@@@@@@@@@@@@@@@@@@@@@@@@@@@@@
\item Week 11 (15 November, 17 November): Framing Problems
\begin{itemize}
\item Context: Stone, Part III
\end{itemize}

%@@@@@@@@@@@@@@@@@@@@@@@@@@@@@@@@@@@@@@@@
\item Week 12 (22 November, 24 November): Solutions
\begin{itemize}
\item Context: Stone, Part IV
\end{itemize}

%@@@@@@@@@@@@@@@@@@@@@@@@@@@@@@@@@@@@@@@@
\item Week 13 (29 November, 1 December): Policy Feedback
\begin{itemize}
\item Research: Mettler, Suzanne. 2002. Bringing the state back in to civic engagement: Policy feedback effects of the GI Bill for World War II veterans. American Political Science Review 96(2): 351-365.
\item Research: \href{https://scholars.org/brief/how-mass-imprisonment-burdens-united-states-distrustful-civic-underclass}{How Mass Imprisonment Burdens the United States with a Distrustful Underclass-Vesla M. Weaver}
\item Listen: \href{https://scholars.org/podcast/government-we-distrust}{SSN: 147: In Government We Distrust-Suzanne Mettler}
\end{itemize}

%@@@@@@@@@@@@@@@@@@@@@@@@@@@@@@@@@@@@@@@@
\item Week 14 (6 November, 8 December): Laws that Govern Lawmaking
\begin{itemize}
%\item Research: Beyond Adversary Democracy-Jane Mansbridge
\item Research: \href{https://canvas.wisc.edu/courses/187920/files/12580467/download}{``Rethinking Representation"} - Jane Mansbridge
\item Listen: \href{https://scholars.org/podcast/citizen-expert}{Citizens’ Initiative SSN 117: The Citizen Expert-John Gastil}
\item Context: \href{https://omny.fm/shows/madsplainers/playlists/podcast/embed?style=artwork}{What is the Local Voices Network?}
\end{itemize}
\textbf{Memo 3 due at midnight December 10}

%@@@@@@@@@@@@@@@@@@@@@@@@@@@@@@@@@@@@@@@@
\item Week 15 (13 December, 15 December): ``Best" Practices, final review
\begin{itemize}
\item Wrap up, final exam review
\end{itemize}
\end{itemize}


\noindent \Large \textbf{Policy on Academic Dishonesty/Plagiarism}\normalsize\\

\noindent Academic dishonesty is broadly defined as submitting work that is not your own without attribution. This is not acceptable in any academic course. I use software tools to detect plagiarism. If you submit written work containing plagiarized material, you will receive a failing grade for the course and be reported to the University.\\


\end{document}
